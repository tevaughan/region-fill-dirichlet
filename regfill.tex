
\documentclass{article}

\usepackage{amsmath}
\usepackage{amssymb} % for \mathbb
\usepackage{times}

\newcommand{\R}{\ensuremath{\mathbb{R}}}
\newcommand{\foreach}{\forall}

\title{%
Using Laplace's Equation to Fill a Bounded Region of a Monochromatic Image%
}

\author{Thomas E. Vaughan}

\begin{document}

\maketitle

\begin{abstract}

   Let $M$ be the set of pixels in a monochromatic image.  Consider any proper
   subset $S \subsetneq M$ such that $S$ is sufficiently small as defined
   below.  Let $B_S$ be the boundary of $S$, so that $\foreach q \in B_S$, $q$
   is not in $S$ but shares an edge with some pixel $p \in S$.  Let $T = S \cup
   B_S$.  Then $S$ is sufficiently small if $T \subsetneq M$.  For any pixel $p
   \in M$, let $v_p$ be the original value of $p$.

   I present an algorithm to compute a new value $r_p$ for each pixel $p \in
   S$.  The new value $r_p$ is computed on the basis of the boundary-values
   $\{v_q \: | \: q \in B_S\}$, on the basis of an estimate of the
   high-spatial-frequency noise along the boundary, and as a solution to the
   Dirichlet Problem for each of the boundary-values and the boundary-noise.
   To wit:
   \begin{enumerate}
      \item For each pixel $p \in S$, compute the value $a_p$ of the solution
         to the discrete Dirichlet problem for the boundary-values $\{v_q \: |
         \: q \in B_S\}$.
      \item For each pixel $p \in T$, compute the value $b_p$ of the solution
         to the discrete Dirichlet problem for the boundary-values $\{v_q \: |
         \: q \in B_T\}$.
      \item For each pixel $q$ in $B_S$, compute the value $\sigma_q = |b_q -
         v_q|$, which is taken as an estimate of the standard deviation for
         high-spatial-frequency noise at pixel $q$.
      \item For each pixel $p \in S$, compute the value $\sigma_p$ of the
         solution to the discrete Dirichlet problem for the boundary-values
         $\{\sigma_q \: | \: q \in B_S\}$.
      \item For each pixel $p \in S$, compute the value $r_p = a_p + n_p$,
         where $n_p$ is a random number drawn from the Gaussian distribution
         whose standard deviation is $\sigma_p$.
   \end{enumerate}
   For an image with noise on the pixel values, this approach produces a filled
   region that does not so obviously stand out to the eye as would merely using
   $r_p = a_p$ for the new value of each pixel $p \in S$.

\end{abstract}

\section{Introduction}

In the early part of the Fall Semester of 2018, I was helping my son with his
homework for a course in partial differential equations.  He was solving the
two-dimensional version of the Dirichlet Problem: to find for a closed region
$A$ in the plane a function $f: A \rightarrow \R$ obeying Laplace's Equation,
which can be written for Cartesian coordinates as\footnote{%
   I use ``$\partial_x$'' to represent the operator whose meaning is ``the
   first partial derivative with respect to $x$.'' Similarly, my
   ``$\partial_x^2$'' represents the operator whose meaning is ``the second
   partial derivative with respect to $x$.'' I picked up this notation from Ron
   Kantowski, my professor in the quantum-field-theory course that I took in
   graduate school in the early 1990s at the University of Oklahoma.}
\begin{equation}
   \left[\partial_x^2 + \partial_y^2\right] f(x,y) = 0.
\end{equation}
Along the boundary of $A$, the function's value is known ahead of the solution.

While thinking about his problem, I was reminded of a problem that I had faced
in a digital-image-processing course that I took when I was in college,
probably around 1990, as I was finishing my undergraduate studies.

\end{document}

