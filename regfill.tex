
\documentclass{article}

\usepackage{amsmath}
\usepackage{times}

\title{%
Using Laplace's Equation to Fill a Bounded Region of a Monochromatic Image%
}

\author{Thomas E. Vaughan}

\begin{document}

\maketitle

\begin{abstract}

   I present an algorithm for filling in a bounded region $P$ of a
   monochromatic image.
   \begin{enumerate}
      \item The pixel-values on $P$'s boundary $B$ are used to solve, via a
         discretized version of Laplace's equation, for an initial estimate
         $p_i$ of the value of each pixel $i$ in $P$.  This is the solution to
         a discrete version of the Dirichlet problem.
      \item The Dirichlet problem is then solved for the region $Q$, formed by
         enlarging $P$ slightly, so that $Q$ includes $B$; this produces the
         value $q_j$ of each pixel $j$ in $Q$.
      \item For each pixel $k$ in $B$, the value $\sigma_k = |b_k - q_k|$,
         where $b_k$ is the value of pixel $k$, is taken as an estimate of the
         standard deviation for high-spatial-frequency noise at pixel $k$.
      \item The Dirichlet problem is then solved again for the region $P$, but
         now with $\sigma_k$ as the value of pixel $k$ in the boundary $B$; the
         result is the value $\sigma_i$ for each pixel $i$ in $P$.
      \item The final value for pixel $i$ in $P$ is $r_i = p_i + n_i$, where
         $n_i$ is a random number drawn from the Gaussian distribution whose
         standard deviation is $\sigma_i$.
   \end{enumerate}
   For an image with noise on the pixel values, this approach produces a filled
   region that does not so obviously stand out to the eye as would merely using
   $p_i$ for the value of each pixel $i$ in the filled region.

\end{abstract}

\section{Introduction}

\end{document}

